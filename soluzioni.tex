\documentclass[a4paper]{article}
\usepackage[T1]{fontenc}
\usepackage[utf8x]{inputenc}
\usepackage[italian]{babel}
\usepackage{enumerate}
\begin{document}
	\title{Soluzioni esercizi gruppi}
	\maketitle
	
	\section{Gruppi: assiomi proprietà generali}
	\subsection{Assiomi}
	\begin{enumerate}[(a)]
	\item[(c)] supponiamo che esistano unità destra $e_s$ e sinistra $e_d$
	allora si può dire $e_s = e_se_d = e_d$
	\item[(d)]	come sopra se esistono inverso sinistro $a_s^{-1}$ e destro $a_d^{-1}$
	allora si può dire $a_s^{-1} = a_s^{-1} a a_d^{-1} = a_d^{-1}$
	\end{enumerate}
	
	\section{ Gruppi di Lie e Algebre di Lie}
	
	\section{Rappresentazioni del gruppo e delle Algebre}
	
	\section{Gruppi di omotopia}
	
	\section{$SU(2)$, $SO(3)$, $SU(3)$}
	
	\section{$SO(4)$ e il gruppo di Lorentz proprio, $SO(3,1)$}
	
	\section{Radici e pesi}	
	
\end{document}